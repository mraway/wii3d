\documentclass[10pt,twocolumn,letterpaper]{article}

\usepackage{cvpr}
\usepackage{times}
\usepackage{epsfig}
\usepackage{graphicx}
\usepackage{amsmath}
\usepackage{amssymb}

% Include other packages here, before hyperref.

% If you comment hyperref and then uncomment it, you should delete
% egpaper.aux before re-running latex.  (Or just hit 'q' on the first latex
% run, let it finish, and you should be clear).
\usepackage[pagebackref=true,breaklinks=true,letterpaper=true,colorlinks,bookmarks=false]{hyperref}


\cvprfinalcopy % *** Uncomment this line for the final submission

\def\cvprPaperID{****} % *** Enter the CVPR Paper ID here
\def\httilde{\mbox{\tt\raisebox{-.5ex}{\symbol{126}}}}

% Pages are numbered in submission mode, and unnumbered in camera-ready
\ifcvprfinal\pagestyle{empty}\fi
\begin{document}

%%%%%%%%% TITLE
\title{Review: Markerless Tracking using Planar Structures in the Scene}

\author{Wei Zhang\\
Department of Computer Science\\
University of California, Santa Barbara\\
{\tt\small wei@cs.ucsb.edu}\\
% For a paper whose authors are all at the same institution,
% omit the following lines up until the closing ``}''.
% Additional authors and addresses can be added with ``\and'',
% just like the second author.
% To save space, use either the email address or home page, not both
%\and
%Second Author\\
%Institution2\\
%First line of institution2 address\\
{\small\url{http://www.cs.ucsb.edu/~wei}}
}

\maketitle
\thispagestyle{empty}

%%%%%%%%% ABSTRACT
\begin{abstract}
In this paper, the author
proposed a taxonomy for dense two-frame stereo correspondence algorithms.
This taxonomy can be used to highlight the 
most important features of existing stereo algorithms,
and to study important algorithmic components in isolation. 
A software framework of stereo
matching algorithm components are implemented to test the performance 
of these algorithms.
And they also performed an exhaustive experimental investigation in
order to assess the impact of the different algorithmic components. 

Since this paper is proposing a survey rather than a new algorithm, 
and because I'm still a beginner in stereo vision, 
in this review I will mainly focused on summarizing the taxonomy proposed by the author.
By doing this I can describe an overview
of the essential idea and structure of most dense stereo algorithms
that being covered in this paper.
\end{abstract}

%%%%%%%%% BODY TEXT
\section{Introduction}
In stereo correspondence, it is difficult to measure the progress of this field
because a lack of a common quantitative method. There are two previous 
comparative papers that focused on the performance of
sparse feature matchers (Hsieh et al. 1992, Bolles et al. 1993), 
and two recent papers (Szeliski 1999,
Mulligan et al. 2001) that compared the performance of
several popular algorithms, but did not provide a detailed taxonomy 
or a complete coverage of algorithms.
This paper continued the investigations begun by Szeliski and Zabih (1999).

However, this paper is not an simple attempt to provide a taxonomy of stereo algorithms 
(otherwise it might not get published on IJCV).
The taxonomy they proposed is designed to identify the individual components 
and design decisions within published algorithms. 
The goal of such effort is to finally create a standard test-bed 
for the quantitative evaluation of dense stereo algorithms and 
thus help researchers to develop new and better algorithms.
%------------------------------------------------------------------------
\section{The Taxonomy}
Since the goal of this work is to compare a large number of methods within one common framework,
several general assumptions about the physical world and the
image formation process are needed. The first common
assumption is that algorithms will mostly deal with Lambertian surfaces whose appearance does not vary with
viewpoint. Another important assumption is physical world consists of piecewise-smooth surfaces.
Furthermore, algorithms in this framework are assumed to be given 
a pair of rectified images as input, and the expected output is the disparity space
image, which is an univalued function in disparity
space $d(x, y)$ that best describes the shape of the surfaces in the scene.

Then the taxonomy can be proposed based on the observation that 
many stereo algorithms generally perform some or all of the following four steps 
(actual sequence of steps taken depends on the specific algorithm).
%------------------------------------------------------------------------
\subsection{Matching cost computation}
In this step, the problem of matching cost computation is standardized as:
Calculate the matching cost values over all pixels and 
all disparities form the initial disparity space image
${C}_{0}(x, y, d)$. 

In its implementation, this paper evaluated many algorithms and additional
techniques, such as 
{\em squared intensity differences} (most commonly used), 
{\em cross-correlation}, 
{\em robust matching score}, 
{\em fractional disparity evaluation}, 
{\em sampling insensitive interval-based matching criterion}.
%-------------------------------------------------------------------------
\subsection{Aggregation of cost}
There are many local and window-based methods 
aggregate the matching cost by summing or averaging over a
support region in the DSI $C(x, y, d)$. 
A support region can be either two-dimensional at a fixed
disparity (favoring fronto-parallel surfaces), 
or three-dimensional in $x-y-d$ space (supporting slanted
surfaces).
There's also different method of aggregation, such as 
iterative diffusion, in which an aggregation (or averaging) operation
is implemented by repeatedly adding to each pixel's cost 
the weighted values of its neighboring
pixels' costs.

This paper implemented two commonly used aggregation methods, 
{\em box filter} and {\em binomial filter}.
{\em Separable square min-filter}, 
{\em cascaded effect of a box-filter} and an
{\em equal-sized min-filter} can be added afterwards if 
the algorithm need the effect of shiftable windows.
%-------------------------------------------------------------------------
\subsection{Disparity computation and optimization}
For this component, the optimization methods are summarized as 5 algorithms, 
all of which minimize the same 
objective function, enabling a more meaningful comparison of their performance.
\begin{itemize}
\item {\em Winner-take-all} (WTA): Simply picks the 
lowest (aggregated) matching cost as the selected
disparity at each pixel.
\item {\em Dynamic programming} (DP): A global 
optimization technique, works by computing the minimum-cost
path through each $x-d$ slice in the DSI.
\item {\em Scanline optimization} (SO): A simple approach 
designed to assess different smoothness terms. It operates on 
individual $x-d$ DSI slices and optimizes one scanline at a time, 
but it is asymmetric and does not utilize visibility or ordering constraints.
\item {\em Simulated annealing} (SA): A classic optimization method, 
in this implementation it supports both the Metropolis variant (where downhill
steps are always taken, and uphill steps are sometimes taken), and the Gibbs Sampler, which
chooses among several possible states according to the full marginal distribution).
\item {\em Graph cut} (GC): Implements the a-b swap move algorithm described
in (Boykov et al. 1999, Veksler 1999).
\end{itemize}

%-------------------------------------------------------------------------

\subsection{Refinement of disparities}
They only discussed the sub-pixel refinement of disparities.

%-------------------------------------------------------------------------

\section{Experiment}
In the experiment section they made comprehensive evaluation 
for individual building blocks of stereo algorithms, and there are
much more detailed result available on their website. 
The experiments discussed in this paper demonstrated 
the limitations of local methods, and have assessed the value of different global
techniques and their sensitivity to key parameters.
%-------------------------------------------------------------------------
\subsection{Test data}
The evaluation requires data sets that either have a
ground truth disparity map, or a set of additional 
views that can be used for prediction error test (or
preferably both. Each image sequence consists of 9 images, 
taken at regular intervals
with a camera mounted on a horizontal 
translation stage, with the camera pointing perpendicularly
to the direction of motion. First all of the sequences 
are made up of piecewise planar objects, 
then use a direct alignment technique on each
planar region (Baker et al. 1998) to estimate 
the affine motion of each patch. The horizontal
component of these motions is then 
used to compute the ground truth disparity.

\subsection{Quality metrics}
They actually measured three quality measures based
on known ground truth data:
\begin{itemize}
\item RMS (root-mean-squared) error (measured in disparity units) between the computed depth
map ${d}_{C}(x, y)$ and the ground truth map ${d}_{T}(x, y)$.
\item Percentage of bad matching pixels.
\item Use the color images and disparity maps 
to predict the appearance of other views (Szeliski 1999), 
then compare with the real image.
\end{itemize}

\section{Contribution and limitations}
I think the major contribution of this paper are:
\begin{itemize}
\item It has an comprehensive overview on the state 
of art in dense stereo algorithms, this could be very 
helpful for people who wants to enter this area.
\item It provides a taxonomy of existing stereo
algorithms, with standard data sets for testing. This is valuable since
the ground truth of image is not easy to get.
\item Under the taxonomy and dataset, the author implemented a lot of 
dense stereo vision algorithms and evaulated them.
\item Finally it proposes a framework for researchers 
to modularize their algorithms, thus makes the results to 
be more fairly comparable, and future algorithm evaulation becomes easier.
\end{itemize}

However, from my point of view, the author's methodology also has some limitations 
(but since I'm not an expert in this area, these might be wrong):
\begin{itemize}
\item First, many published methods include special features and post-processing
steps to improve the results, is it fair to ignore these features and only 
compare the basic version? Maybe these features can affect the results of some algorithms
seriously, especially when I found that the results of some algorithms on the website
are too bad to get published.
\item Second, only algorithms in certain kind can be evaluated in this framework. 
So I have question in how well can later algorithms be fitted into it? But since
there are many algorithms implemented, this is already a great work.
\end{itemize}
%-------------------------------------------------------------------------
\section{Summary}
All in all, this is a very good paper. It is very well-written, covers major dense two-frame
stereo correspondence algorithms at the state of art, and still keep updating on the website. 
The analysis, comparison, and evaluation of these algorithms are comprehensive and impressive.
This paper might be a little difficult for beginners since it requires some prior experiences
, but it is an important tool and perfect reference for researchers who are working in this area.

%-------------------------------------------------------------------------
\end{document}
