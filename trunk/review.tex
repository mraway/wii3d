\documentclass[10pt,twocolumn,letterpaper]{article}

\usepackage{cvpr}
\usepackage{times}
\usepackage{epsfig}
\usepackage{graphicx}
\usepackage{amsmath}
\usepackage{amssymb}

% Include other packages here, before hyperref.

% If you comment hyperref and then uncomment it, you should delete
% egpaper.aux before re-running latex.  (Or just hit 'q' on the first latex
% run, let it finish, and you should be clear).
\usepackage[pagebackref=true,breaklinks=true,letterpaper=true,colorlinks,bookmarks=false]{hyperref}


\cvprfinalcopy % *** Uncomment this line for the final submission

\def\cvprPaperID{****} % *** Enter the CVPR Paper ID here
\def\httilde{\mbox{\tt\raisebox{-.5ex}{\symbol{126}}}}

% Pages are numbered in submission mode, and unnumbered in camera-ready
\ifcvprfinal\pagestyle{empty}\fi
\begin{document}

%%%%%%%%% TITLE
\title{Review: Markerless Tracking using Planar Structures in the Scene}

\author{Wei Zhang\\
Department of Computer Science\\
University of California, Santa Barbara\\
{\tt\small wei@cs.ucsb.edu}\\
% For a paper whose authors are all at the same institution,
% omit the following lines up until the closing ``}''.
% Additional authors and addresses can be added with ``\and'',
% just like the second author.
% To save space, use either the email address or home page, not both
%\and
%Second Author\\
%Institution2\\
%First line of institution2 address\\
{\small\url{http://www.cs.ucsb.edu/~wei}}
}

\maketitle
\thispagestyle{empty}

%%%%%%%%% ABSTRACT
\begin{abstract}
In this paper, the author describe a vision-based position tracker
 for augmented reality that operates in unstructured environment.
 This simplifies the general camera-tracking problem by only requires
 one or more planar surface visable in the scene, which is a common
 special case. In addition, the tracked plane represents a natrul
 reference frame so that the alignment of real and virtual coordinate
 systems is simplified as well. The author also provide an
 implementation of a practical and reliable tracker to show its validity.
Their method grasps the key of markerless tracking: detect and track
 something in the image as an alternative of markers. Therefore finding
 a planar surface may be a pretty straightforward solution. However,
 their method is also limited by this point. Futhermore, their implementation
 is not completely automatic, it requires the user to choose 4 points
 of a rectangular on the plan manually during the initialization.
\end{abstract}

%%%%%%%%% BODY TEXT
\section{Introduction}
\subsection{The Problem}
Estimating the pose and movement of a camera (virtual or real)
in which some augmentation takes place is one of
the most important parts of an augmented reality
(AR) system. Legacy vision-based trackers are based on
tracking of markers. The use of markers increases
robustness and reduces computational requirements.
However, their use can be very complicated,
as they require certain maintenance. Therefore, directly using
 the scene features for tracking is desirable.

%-------------------------------------------------------------------------
\subsection{Some Previous Work}
Now the problem turns to be which feature to use and how. 
The most common approach is model-based tracking, which 
exploit some structures in the scene 
that their measurements are already known. This method provide 
precise result, but is greatly limited by requiring the manual 
intervention to construct the model.\\

Another technology that provides general, accurate registration
 is known as structureand-motion estimation, or move-matching. 
It simultaneously estimate camera motion, and
the 3D structure of the imaged scene. 
The major limitation is that it needs a lot computation 
and a bunch of adjustments, 
so that cannot be used in real-time application.

\subsection{This Paper}
In this paper, the author try to provide a markerless solution with only slight
limitation on the type of scene. Some previous works had already done 
something similar such as using sets of parallel or orthogonal lines 
in the scene to determine the 3D reference frame. So their idea of 
using planar surface doesn't contribute too much. But they propose 
an efficient algorithm and a very clear framework to
track the planar surface and do camera recovery fast and reliably.\\

In general, I think this paper is well-written and easy
 to read. They first give a brief introduction to the problem 
and previous works, all with cons and pros. Then they briefly 
introduce the main idea of this work, and give a algorithm summary 
to show how it looks like. After that they describe every part 
of the algorithm in detail, the way that they explain is pretty easy 
to understand, but they do forget to add the equation numbers. Then they 
show us an implementation and some results as an example, and finally 
they discuss the advantage and limitation of this method.


%------------------------------------------------------------------------
\section{Discussion of the Approach}
\subsection{Main Idea}
The key idea of this paper is to track the planar surface in sequential images. 
By matching the interesting points between two images, the RANSAC algorithm gives out 
the planar homography between them. Therefore, if the planar homography between a planar 
surface in the world and in one of these images are determined, then problem is solved. 
This paper suggests to do this manually on the first image during initialization, 
by specifying a rectangular on a plane in the image.
%-------------------------------------------------------------------------
\subsection{Pros}
There are several points that I like in this paper:
\begin{enumerate}

\item The choice of planar surface is a good idea, because such plane exists 
in most scenes that contain man-made objects.

\item They let the user to select a plane, and make it 
the XY plane in the world coordinate system. So all points on this plane have
{\em z} equals zero, this makes future computation easier.

\item They give out a comprehensive explanation on how to compute the 
planar homography transformation matrices, as well on other steps, 
which makes their algorithm easy to repeat.

\item They argue that the best homography for the whole image(found by RANSAC) 
will correspond to the homography for the largest plane in the image. 
I'm not sure how robust it is, but if it's true then the hand-off precedure 
could be easier to implement.

\item The demos that they show is pretty impressive.

\end{enumerate}
%-------------------------------------------------------------------------
\subsection{Cons}
But there are more things that I have question with:
\begin{enumerate}

\item There have no equation numbers in the paper, that's a bad habit 
since this paper has more than 20 equations.

\item In {\em section 2.3}, they mention a little bit on 
some related works that detect parallel or orthogonal lines in the image 
to improve AR, and then skip them rapidly. 
I think these works are much more related to this paper, compare to what they 
have discussed in {\em section 2.1} and {\em section 2.2}. I think they should 
not intend to avoid the comparison with other competitive works.

\item In {\em section 3.6} they state that the '{\em hand-off}' procedure is easy, 
without giving enough details on how. And it becomes even more confusing when 
they admit in the conclusion that their system '{\em might fail}', 
'{\em when the tracked plane goes out of view}'. The problem here is that they doesn't 
explain how to choose the new plane and find three points on it. Is it done by RANSAC? 
Can RANSAC provide such robust result? Is there any experiment data to support this?

\item They rely on the RANSAC to compute the planar homography, which is the key part 
of their computation. But I can't find any experimental data analysis about this, especially 
about the accumulative drift. Besides, they doesn't really give a solution to solve this.

\item Finally, they didn't provide experimental analysis of their algorithm, thus I can't tell 
how good or bad this algorithm really is, nor compare to other related works.

\end{enumerate}
%-------------------------------------------------------------------------

\section{Experiment}
There are short demos, but no experiment or data analysis actually.
%-------------------------------------------------------------------------

\section{Conclusion}

This paper propose a simple, fast and reliable solution to markerless camera tracking problem. 
Their major contributions are the mathematical framework for 
uncalibrated plane tracking and camera recovery.\\

However, the limitations of their solution are obvious: 
\begin{itemize}
\item There must be more than one plane in the image, and must be detectable; 
\item The system might fail if the tracked plane move out of view;
\item The system rely on RANSAC to compute the homography between images, 
but sometimes this is not reliable, for example, to track a big plane with many 
small rectangulars which all in the same size and pattern, such as a chessboard;
\item How to do hand-off is still in question.
\end{itemize}

In order to make their argument stronger, the author need to:
\begin{itemize}
\item Give more details at a few points;
\item Provide experimental data analysis;
\item Show comparison with similar works;
\item Finally and most important - add equation numbers.
\end{itemize}

%-------------------------------------------------------------------------
\end{document}
